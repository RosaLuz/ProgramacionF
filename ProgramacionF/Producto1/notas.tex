\documentclass[12pt]{article}

\usepackage[spanish]{babel}
\selectlanguage{spanish}
\usepackage[utf8]{inputenc}


\title{Tutorial breve de los comandos de Bash}
\author{Rosa Luz Zamora Peinado}
\date{05 de Febrero de 2015}


\begin{document}

\maketitle

\section{¿Qué es {\tt bash}?}

Bash (Bourne again shell) es un programa informático cuya función consiste en interpretar órdenes.Es un interpretador de comandos utilizado sobre el sistema operativo Linux. Su función es de mediar entre el usuario y el sistema. stá basado en la shell de Unix y es compatible con POSIX.

\section{Navegación}

Esta sección describe como se puede navegar entre archivos y directorios.

\section{Husmeando en el sistema}

Veremos 3 comandos:\\


\begin{tabular}{|c|l|l|}
\hline
Comando & Descripción & Ejemplo \\
\hline
ls & listar contenido de directorios & ls -al \\ \hline
less & Permite ver el contenido de archivos de texto & less aaa.txt \\ 
\hline
file & Nos regresa el tipo de archivo & file aaa.txt \\
\hline
pwd & Muestra el directorio actual & /home/rlzamorap \\
\hline
\end{tabular} 

\section{Páginas del manual}

El manual es una serie de páginas que te explican cada comando disponible en tu sistema incluyendo su función; te especifica cómo correrlos y que tipo de argumentos reciben.\\


\begin{tabular}{|p{4cm}|p{6cm}|p{4cm}|}
\hline
Comando & Descripción & Ejemplo \\
\hline
Man command & Busca en la página del manual por un comando en particular. & man ls\\ \hline
Man –k (search term) & Hace una búsqueda de las palabras clave para toda página del manual que las contenga. & man -help\\ \hline
\end{tabular}

\section{Manipulando archivos}

Linux organiza su sistema jerárquicamente. Con el tiempo vas a acumular una gran cantidad de información. Es importante crear una estructura de directorios que ayuden a organizar esa información de una manera cómoda.\\

\begin{tabular}{|p{4cm}|p{6cm}|p{4cm}|}
\hline
Comando & Descripción & Ejemplo \\
\hline
mkdir & Crea un directorio & mkdir DIRECTORIO \\ \hline
rmdir & Borra un directorio & rmdir DIRECTORIO \\ \hline
touch & Crear un archivo en blanco & touch ARCHIVO \\ \hline
cp & Copia un archivo o directorio & cp DIRECTORIO \\ \hline
mv & Mueve un archivo o directorio & mv DIRECTORIO Escritorio \\ \hline
rm & Elimina un archivo o directorio & rm DIRECTORIO \\ \hline
cd & Cambia de ubicación, de directorio & cd ESCRITORIO \\ \hline
\end{tabular}

\section{Vi - Editor de texto}

Vi es una linea de comando editora de texto. como ya lo sabes, la línea de comando es ambiente un poco diferente a tu GUI. Vi ha sido diseñado para trabajar con limitaciones y aunque muchos están en desacerdo, en realidad brinda poderosos resultados.\\

\begin{tabular}{|p{4cm}|p{6cm}|p{4cm}|}
\hline
Comando & Descripción & Ejemplo \\
\hline
vi & Editar un archivo & vi TTT \\ \hline
cat & Ver un archivo & cat ARCHIVO \\ \hline
less & Conveniente para ver archivos largos & less ARCHIVO \\ \hline
\end{tabular}

\section{Permisos}

Los permisos de Linux permiten tres cosas que puedes hacer con un archivo: leer,  escribir o ejecutar.\\

r read (leer) - Puedes ver los contenidos del archivo.\\
w write (escribir) - Puedes cambiar los contenidos del archivo.\\
x execute (ejecutar) - Puedes correr o ejecutar el archivo, puede ser un programa.\\

\begin{tabular}{|p{4cm}|p{6cm}|p{4cm}|}
\hline
Comando & Descripción & Ejemplo \\ \hline
chmod & Cambia los permisos de un archivo o directorio. & chmod 100 DIRECTORIO \\ \hline
ls -ld & Muestra los permisos de un archivo o directorio en específico & ls -DIRECTORIO \\ \hline
\end{tabular}

\end{document}



